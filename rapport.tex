%% Time-stamp: <2023-01-09 21:35:45 stefan>

\documentclass[a4paper,swedish]{article}

\usepackage[utf8]{inputenc}
\usepackage[T1]{fontenc}    %% annars fungerar inte klipp-och-klistra från PDF självt
\usepackage{babel}
\usepackage[a4paper,top=2cm,bottom=2cm,left=1.5cm,right=1.5cm]{geometry}
\usepackage{graphicx}
\usepackage[colorlinks=true, allcolors=blue]{hyperref}

%%\usepackage{titlesec}
%%\titleformat{\section}{}{}
\usepackage{tabularx}
\title{Rapport inom el-säkerhet och el-materiel}
\author{Stefan}
\author{Oscar}

\usepackage{fancyhdr}
\setlength{\headheight}{37pt}
% \addtolength{\topmargin}{-22pt}
\fancyhf[LH]{Stefan Niskanen Skoglund\\Oscar}
\fancyhf[RH]{}
\pagestyle{fancy}

\begin{document}

\part{Elsäkerhet}
%% del: säkerhet
\setcounter{section}{1}
\section{Vad är det som påverkar skadeverkan vid strömgenomgång?} \label{sec:strömgenomgång}

Skadan och dess svårighet beror hur på strömstyrkan och hur länge kroppen är utsatt strömpåverkan.
Strömpåverkan kan vara antingen att strömmen passerar mindre eller större del av kroppen eller att kroppen
utsätts för en ljusbåge.

Om man har haft en modelljärnväg, så kan man med två fingrar på samma hand ta på återledaren och mittledaren (Märklins mittledareräl) och känna en svag
vibration. Den strömmen är pga den låga spänningen och hudens höga resistivitet $< 1 mA$
Strömmen ger upphov till vibrationskänslan.

\setcounter{section}{3}
\section{Vilken långsiktig skada uppstår?}\label{sec:permanent_skada}

Nekrosis dvs celldöd. En skada som även har långtidseffekter dvs att exempelvis nervceller
fortsätter att då även en lång tid efter den akuta skadan.

\setcounter{section}{5}
\section{Vilka är skyddsjordledarens PE två huvudfunktioner?}\label{sec:skyddsjordsledarens_funktioner}

Skyddsjordledaren ska ha ge en garanterad förbindelse mellan en apparats hölje och transformatorns nollpunk
så att om apparatens konstruktion
med avseende på elektrisk isolering mellan dess innamäte och höljet skadas, att den resulterande strömmen
kommer att förstöra säkringen inom en kort tidsrymd och därmed bryta strömmen.

Garantera att höljen på apparater inom en anläggning kommer att ha samma potential

\setcounter{section}{7}
\section{Vilka varianter finns det på diazedsäkringen?}\label{sec:diazedvarianter}

Tre olika diametrar av själva säkringskroppen med till var sin passande hylsa.
Hylsorna har olika diameter och gängdimension (Edison-gänga)

\setcounter{section}{9}
\section{Vilken färg på signalpärlan har en 10 A diazedsäkring?}\label{sec:diazed_10A}

Röd

\setcounter{section}{11}
\section{Vad menas med en dvärgbrytares märkkortslutningsbrytförmåga?}\label{sec:RCD_short_interrupt}

Över vilken kortslutningsström dvärgbrytaren inte längre är garanterad att klara av att bryta
den genomgående strömmen.

\setcounter{section}{13}
\section{Vilka klasser delas dvärgbrytare in beroende på deras tröghet?}\label{sec:RCD_classes}

\begin{enumerate}
\item A
\item B
\item C
\end{enumerate}

\setcounter{section}{15}
\section{Vilken klass ska det vara till motorer, mikrovågsugn och centraldammsugare?}\label{sec:RCD_motordrift}

Klass B.

18\setcounter{section}{17}
\section{Vad ska det finnas vid varje central?}\label{sec:vad_ska_finnas_vid_en_central}

En gruppförteckning

\setcounter{section}{19}
\section{Vad vill man åstadkomma med skyddsutjämning?}\label{sec:skyddsutjamning}

Personskydd och i exv djurstallar iom att hästar och kor är mycket känsliga för en liten potentialskillnad
mellan delar av ett golv, att inte skrämma dem med stötar.

\setcounter{section}{21}
\section{Vad ska anslutas till huvudjordningsskenan ?}\label{sec:huvudjordningsskenan}

Elinstallationens skyddsledare dvs husets normala gul-gröna skyddsledare

Inkommande jordtag om huset har ett eget jordtag (praktiskt sett ett krav att ett sådant finns i ett systemd där huvudjordningsskena har använts)

\setcounter{section}{23}
\section{Potentialutjämningsledarens maximala area?}\label{sec:potentialutjamningarea}

Elinstallationens skyddsledare dvs husets normala gul-gröna skyddsledare

Inkommande jordtag om huset har ett eget jordtag (praktiskt sett ett krav att ett sådant finns.)

\setcounter{section}{25}
\section{Vilka är potentialutjämningens två uppgifter?}\label{sec:potentialutjamningsuppgifter}

\setcounter{section}{27}
\section{Hur fungerar en jordfelsbrytare?}\label{sec:RCD_working}

Den jämför igenomgående strömstyrka med den återgående. Är differensen högre än utlösningsvillkoret
(30 mA exv) så bryts strömmen.

\setcounter{section}{29}
\section{I vilka system fungerar en jordfelsbrytare?}\label{sec:RCD_systems}

TN-C och TN-S, skyddsledaren i anläggningen och ledaren för återgångsströmmen måste i anläggningen vara åtskilda.

Det är viktigt att för de belastningar som matas igenom jordfelsbrytaren, att återgångsströmmen från alla de och
bara de går igenom brytaren.

\setcounter{section}{31}
\section{Hur snabbt löser den ut?}\label{sec:RCD_how_fast}

Kortare tid än periodtiden - för 50 Hz  20 ms.

\setcounter{section}{33}
\section{Vilka olika klasser finns det på jordfelsbrytare?}\label{sec:RCD_klasser}

\setcounter{section}{35}
\section{Hur ofta bör man motionera jordfelsbrytaren?}\label{sec:motionering_RCD}

Orsaken är att man vill kontrollera att brytaren löser vid fel, brytaregapet inuti kan vara skadat exempelvis.

Intervallet brukar anges till en gång i halvåret (förslagsvis i samband med övergång till och från sommartid.)

\setcounter{section}{37}
\section{Vilka tre punkter måste man beakta vid säkerställande att en anläggning är
  spänningslös vid ett arbete?}

\begin{enumerate}
\item Frånskiljning, hur stänger vi av strömmen i den här anläggningen ?
\item Hindra tillhoppling, hur garanterar vi att strömmen kopplas på av någon annan eller tar en annan väg ?
\item Spänningsfrihetskontroll, mät att spänningen är borta (och kontrollera att mätinstrumentet är helt både innan och efteråt.)
\end{enumerate}

\setcounter{section}{39}
\section{Vilka personer är inblandade när Heta arbeten utförs?}\label{sec:heta_arbeten_roller}

\begin{enumerate}
\item Tillståndsansvarige (beställaren av arbetet)
\item den som utför arbetet
\end{enumerate}

Tillståndsansvariges tillstånd måste om arbetet är bedömt som brandfarligt ge sitt tillstånd till att påbörja arbetet.

\setcounter{section}{41}
\section{Vad är grundtanken med utrustningar som är placerade i riskområden?}

\setcounter{section}{43}
\section{Vilka klasser klassas brandsläckare enligt den europeiska standarden SS EN3?}

\begin{enumerate}
\item A -- inriktad mot brand i fibrösa material exv trä och tyg
\item B -- brinnande vätskor exv bensin, fotogen
\item C -- gasformiga bränslen
\item D -- avsedd för brand i magnesium och andra reaktiva metaller.
\item F -- avsedd för brand i oljor
\end{enumerate}

\setcounter{section}{45}
\section{Vilken släckningsutrustning är bra att ha hemma i sin bostad?}\label{sec:slackutrustning}

\begin{enumerate}
\item En brandfilt för kvävning av branden (exv om frityrolja överhettas eller en brand på bord
  efter att ljus har vält
\item Pulverbrandsläckare med en volym av minst 6 kg rekommenderas av försäkringsbolagen, räddningstjänsterna och MSB. Filtens fördel är att den inte smutsar ner på samma vis.
\end{enumerate}

En trasmatta iom att den är kompakt fungerar i nödfall.

\setcounter{section}{47}
\section{Vem är det som bär det fulla ansvaret för de installationsarbeten som anställda utför?}\label{sec:ansvar_elarbeten}

Företagsledningen -- denna ska:
\begin{enumerate}
\item Implementera och tillse att företaget följer sitt egenkontrollprogram
\item Ha en juridisk uppgörelse av någon typ med en auktoriserad elinstallatör där auktorisationen
  gäller för företagets arbetsområde
\item Den auktoriserade personen stöttar företagsledningen i dess arbete med egenkontrollprogrammet
\end{enumerate}

\setcounter{section}{49}
\section{Vad måste alla personer som utför elinstallationsarbete ha?}

\setcounter{section}{51}
\section{Vad innebär kravet på registrering för elsäkerhetsverket?}

\setcounter{section}{53}
\section{Vad måste varje elinstallationsföretag ha?}\label{sec:designated_person}

\begin{enumerate}
\item Ett egenkontrollprogram
\end{enumerate}

\setcounter{section}{55}
\section{Vilket ansvar har anläggningsinnehavaren enligt elsäkerhetsverket?}\label{sec:innehavarens_ansvar}

Allt.

\setcounter{section}{57}
\section{Vad är skillnaden mellan en lågspänningsanläggning och en högspänningsanläggning?}

Spänningen, allt med en spänning $0 < U <1000$ V räknas som lågspänning, från 1 kV högspänning.

\setcounter{section}{59}
\section{Vad menas med utsatt del?}

Begreppet definieras i ELSÄK-FS 2008:1 kap 1 paragraf 3 definitioner som:

\begin{center}
  \begin{tabular}{|l p{10cm}|}
    utsatt del & för beröring åtkomlig ledande del av elektrisk materiel
                 som normalt inte är spänningssatt men som vid fel på
                 grundisoleringen kan bli spänningssatt,
  \end{tabular}
\end{center}

\setcounter{section}{61}
\section{Vilka skydd skyddar mot direkt och indirekt beröring?}

\setcounter{section}{63}
\section{Vilken myndighet utarbetar föreskriften ELSÄK-FS 2008:1?}

\setcounter{section}{65}
\section{Hur många kapitel innehåller ELSÄK-FS 2008:1?}

7 stycken.

\setcounter{section}{67}
\section{Vilka elinstallationsarbeten omfattas inte av elsäkerhetslagen under förutsättning att elinstallationsarbetet inte utförs i potentiellt explosiva miljöer.}

\begin{itemize}
\item Byte av strömbrytare (elkopplare) för högst 16 A 400 V
\item Byte av anslutningsdon (vägguttag), även för detta gäller begränsningan 16 A 400 V
\item byte av ljusarmaturer i bostäder
\item Arbeten i ett SELV-system dvs där det finns skyddstransformatorer och systemspänningen inte är högre än 50 V
  och att maximal ström är begränsad via säkring till som mest 10 A.
\item Installation av en värmekabel eller värmefolie för högst 50 V och om den är ansluten till skyddsklenspänning (en skyddstransformator isolerar systemet mot högre spänningar)
\item Kopplingsarbeten i lab-miljö
\end{itemize}

\setcounter{section}{68}
\section{Vad kallas den metod som EIO tagit fram för besiktning av elanläggningar i bostad?}

IN Eltest. IN iom att EIO har bytt namn till Installatörsföretagen.

\part{Elmaterial}

\setcounter{section}{0}
\section{Hur vet man vilket elmateriel man ska välja till en elinstallation?}

Kablar väljs antingen i förväg i ritningsskedet av konstruktör eller av
elektriker (eller eventuellt dennes arbetsledning.)

Konstruktören behöver bland annat ha uppgifter från nätägaren om vissa karakteristika
för det lokala nätet.

Installatören väljer kapslingar, kopplingsdon (vägguttag) och brytare efter vilket elmiljö (område)
de ska vara i och deras användning. Olika elmiljöer ställa olika minimum-krav på materialets IP-klassning.

\setcounter{section}{2}
\section{Vad betyder IP?}

Den grad av skydd mot direkt och indirekt kontakt med strömförande delar som en kapsling ger.

\setcounter{section}{4}
\section{Vad anger första siffran?}

\setcounter{section}{6}
\section{Vilken kapslingsklass ska elmateriel om tillhör ”torra utrymmen” ha?}

\setcounter{section}{8}
\section{Vilken kapslingsklass ska elmateriel ha i vissa delar av badrum, grovkök och
  tvättstugor, där risk för överspolning finns?}

I de elmiljöerna där det bedöms att utrustningen kan bli direkt överspolade med vatten, krävs det minst IPx4.

\setcounter{section}{10}
\section{Av vilka delar består en kabel?}

\begin{enumerate}
\item ledare
\item separat isolering för varje ledare, biledare är ett undantag
\item mantel/ytterskal hölje för kabeln
\end{enumerate}

\setcounter{section}{12}
\section{Varför förser man kablar med en skärm av tex aluminiumfolie?}

Kontroll av magnetiska fält runt om kablen

\setcounter{section}{14}
\section{Vad betyder EKK?}

Enkelkardelig dvs varje ledare består av en massiv ledare

\setcounter{section}{16}
\section{Vad anger första bokstaven?}

\setcounter{section}{18}
\section{Tredje bokstaven?}

\setcounter{section}{20}
\section{Vad gäller i den nya CPR-klassningen för brandskydd?}

Innan Juli 2017 omfattade CPR-klassningen enbart risken för brandspridning.

Den nuvarande CPR-klassningen differentierar mellan:

\begin{center}
  \begin{enumerate}
  \item Brandspridning
  \item Värmeutveckling
  \item Rökutveckling
  \item Droppbildning
  \item Syrahalt (PVC omvandlas i branden till HCl, saltsyra.)
  \end{enumerate}
\end{center}

\setcounter{section}{22}
\section{Till vad används kraftkablar?}

Överföring och distribution av el, minsta area i varje tråd är $2,5 mm^2$
\setcounter{section}{24}
\section{Till vad används anslutningskablar?}

Anslutning av apparater till sina matningar. Den stora skillnaden jämfört med kraftkabel är anslutningskablars
tålighet mot vibrationer och böjning.

\setcounter{section}{26}
\section{Till vad används signalkablar?}

Används i diverse styr-tillämpningar där strömmarna är låga.
Larmsystem och reglersystem är exempel.

\setcounter{section}{28}
\section{Vilken är biledaren funktion?}

\setcounter{section}{30}
\section{Vad undviker man genom att ansluta biledaren till skyddsjord i ena änden på
kabeln?}

\setcounter{section}{32}
\section{Hur betecknar man dessa materielklasser?}

\end{document}
